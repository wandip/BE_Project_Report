%This is a very basic  BE PROJECT PRELIMINARY template.

%############################################# 
%#########Author :  PROJECT###########
%#########COMPUTER ENGINEERING############


\documentclass[oneside,a4paper,12pt, times]{book}
\usepackage[left=1.5in, right=1in, top=1in, bottom=1.25in]{geometry}
\onehalfspacing
\setlength{\parskip}{12pt}
%\usepackage{showframe}
%\hoffset = 8.9436619718309859154929577464789pt
%\voffset = 13.028169014084507042253521126761pt
\usepackage{fancyhdr}

\fancypagestyle{plain}{%
  \fancyhf{}
  \fancyfoot[CE]{Pune Institute of Computer Technology, Department of Computer Engineering 2016-17}
  \fancyfoot[C]{\thepage}
}
\pagestyle{plain}
\renewcommand{\headrulewidth}{0pt}
\footskip = 0.625in
\cfoot{}
\rfoot{}

\usepackage[]{hyperref}
\usepackage{tikz}
\usetikzlibrary{arrows,shapes,snakes,automata,backgrounds,petri}
\usepackage{titlesec}
\usepackage{tabularx}

\usepackage[nottoc,notlot,notlof,numbib]{tocbibind}
\usepackage[titletoc]{appendix}
\usepackage{titletoc}
\renewcommand{\appendixname}{Annexure}
\renewcommand{\bibname}{References}

\setcounter{secnumdepth}{5}

\usepackage{float}
\usepackage{subcaption}
\usepackage{multirow}

\usepackage[ruled,vlined]{algorithm2e}

\begin{document}

\setlength{\parindent}{0mm}
\begin{center}

 \vspace*{1\baselineskip}
{\bfseries A PRELIMINARY REPORT ON \\}
 \vspace*{2\baselineskip}
{\bfseries \fontsize{16}{12} \selectfont  Comprehensive Developer Assistant (CODA) \\ \vspace*{2\baselineskip}}
{\fontsize{12}{12} \selectfont SUBMITTED TO THE SAVITRIBAI PHULE PUNE UNIVERSITY, PUNE \\
IN THE PARTIAL FULFILLMENT OF THE REQUIREMENTS \\
FOR THE AWARD OF THE DEGREE \\ 
\vspace*{2\baselineskip}
OF\\

\vspace*{2\baselineskip}}
{\bfseries \fontsize{14}{12} \selectfont \mbox{BACHELOR OF ENGINEERING(Computer Engineering)} \\
\vspace*{1\baselineskip}} 
{\bfseries \fontsize{14}{12} \selectfont SUBMITTED BY \\ 
\vspace*{1\baselineskip}} 
Shantanu Jain  \hspace{25 mm} Exam No : B150054307\\
Tanmay Singhal \hspace{25 mm} Exam No : B150054452 \\
Tejas Srivastava   \hspace{25 mm} Exam No : B150054460\\
Dipak Wani \hspace{25 mm} Exam No : B150054476\\
\vspace*{2\baselineskip}

\vskip 1cm
\includegraphics[width=100pt]{college_logo.jpg} \\
\vskip 1cm
{\bfseries \fontsize{15}{12} \selectfont 
DEPARTMENT OF COMPUTER ENGINEERING\\
}
{\bfseries \fontsize{14}{12} \selectfont{
\vspace*{1\baselineskip}
PUNE INSTITUTE OF COMPUTER TECHNOLOGY \\
\vspace*{1\baselineskip}
DHANKAWADI, PUNE - 43 \\
\vspace*{1\baselineskip}

}
\end{center}

\newpage

{\bfseries \fontsize{14}{12} \selectfont \centerline{SAVITRIBAI PHULE PUNE UNIVERSITY}
\vspace*{1\baselineskip}} 
{\bfseries \fontsize{14}{12} \selectfont \centerline{2018-2019}
\vspace*{1\baselineskip}} 
\begin{figure}[ht]
\centering
\includegraphics[width=100pt]{college_logo.jpg}
\end{figure}





{\bfseries \fontsize{16}{12} \selectfont \centerline{CERTIFICATE} 
\vspace*{2\baselineskip}} 

\centerline{This is to certify that the Project Entitled}
\vspace*{.5\baselineskip} 


{\bfseries \fontsize{14}{12} \selectfont \centerline{Comprehensive Developer Assistant (CODA)}
\vspace*{0.5\baselineskip}}

\centerline{Submitted by}
\vspace*{0.5\baselineskip} 
\centerline{Shantanu Jain  \hspace{25 mm} (B150054307)}
\centerline{Tanmay Singhal \hspace{25 mm} (B150054452)}
\centerline{Tejas Srivastava \hspace{25 mm}    (B150054460)}
\centerline{Dipak Wani \hspace{25 mm} (B150054476)}

is a bonafide work carried out by students under the supervision of \textbf{Prof. M.S. Chavan} and it
is submitted towards the partial fulfillment of the requirement of \textbf{Bachelor of Engineering (Computer Engineering)}.\\
\vskip 1cm
\bgroup
\def\arraystretch{0.7}
\begin{tabular}{c c }
\textbf{Prof. M.S. Chavan} &  \hspace{50 mm} \textbf{Dr. R. B. Ingle} \\								
Internal Guide   &  \hspace{50 mm} H.O.D \\
Dept. of Computer Engg.  &	\hspace{50 mm}Dept. of Computer Engg.  \\
\end{tabular}
%}
\begin{center}
%\fontsize{12}{18}\selectfont 
{
\vskip 1cm
\textbf{Dr. P. T. Kulkarni}\\
Principal\\
Pune Institute of Computer Technology  
}
\end{center}
\vskip 1cm
Signature of Internal Examiner \hspace{40 mm}\mbox{Signature of External Examiner}

\newpage

%\pictcertificate{TITLE OF BE PROJECT}{Student Name}{Exam Seat No}{Guide Name}
\setcounter{page}{0}

\frontmatter
\cfoot{PICT, Department of Computer Engineering 2018-19 \\ \\ \thepage}
%\rfoot{\thepage}
\pagenumbering{Roman}
%\pictack{BE PROJECT TITLE}{Guide Name}


{  \newpage {\bfseries \fontsize{14}{12} \selectfont \centerline{Acknowledgments} 
\vspace*{2\baselineskip}} \setlength{\parindent}{11mm} }
{ \setlength{\parindent}{0mm} }
%Please Write here Acknowledgment.Example given as\\
\textit{It gives us great pleasure in presenting the preliminary project report 
on {\bfseries \fontsize{12}{12} \selectfont `{\bfseries \fontsize{14}{12} \selectfont {Comprehensive Developer Assistant (CODA)}
}'}.}


 \textit{We would like to take this opportunity to thank our internal guide
 \textbf{Prof. M.S. Chavan} for giving us all the help and guidance we needed. We are really grateful to him for his kind support. His valuable suggestions were very helpful.} 
\\
\\
 \textit{We are also grateful to \textbf{Dr. R. B. Ingle}, Head of Computer
 Engineering Department, PICT for his indispensable
 support along with his guidance.}

\\
\textit{In the end our special thanks to the college for
providing various resources such as  laboratory with all needed software platforms,
continuous Internet connection, for our Project.}
\vspace*{3\baselineskip} \\
\begin{tabular}{p{8.2cm}c}
&Shantanu Jain\\
&Tanmay Singhal\\
&Tejas Srivastava\\
&Dipak Wani\\
&(B.E. Computer Engg.)
%}
\end{tabular}



{  \newpage {\bfseries \fontsize{14}{12} \selectfont \centerline{Abstract} 
\vspace*{2\baselineskip}} \setlength{\parindent}{11mm} }
{ \setlength{\parindent}{10mm} }

 \-\hspace{0.5in}\textit{Robots and AI based systems are now slowly taking over many sectors such as support,
industrial automation, etc. Our aim is to develop one such intelligent engine, which will
allow users to give commands in natural English to perform a particular task/s on the client
machine, which would then be analyzed and converted into appropriate terminal
commands/code snippet generation using natural language Processing and Machine
Learning. System can be integrated with VCS like GitHub, Project management application
like Slack. The system will be able to constantly evolve and train itself to handle variations
in user input. Such a system finds useful application in a Virtual Intelligent Assistant,
which can be based on remote access and can be handled easily by naive users as well.
principles, to run on the client device}

Keywords : NLP, Deep Learning, Text Mining
\\ 
\\
\-\hspace{0.5in}

% \maketitle
\tableofcontents
\listoffigures



\mainmatter



\titleformat{\chapter}[display]
{\fontsize{16}{15}\filcenter}
{\vspace*{\fill}
 \bfseries\LARGE\MakeUppercase{\chaptertitlename}~\thechapter}
{1pc}
{\bfseries\LARGE\MakeUppercase}
[\thispagestyle{empty}\vspace*{\fill}\newpage]







\setlength{\parindent}{11mm}
\chapter{Introduction}

\section{Motivation}
\-\hspace{0.5in}Para1
\begin{figure}[h]
    \centering
    \includegraphics[width=400pt,height=250pt]{motivation.png}   
    \caption{Figure title}
    \label{fig:my_label}
\end{figure}

Para 2
\begin{figure}[h]
    \centering
    \includegraphics[width=400pt,height=250pt]{motivation_.png}   
    \caption{Figure title}
    \label{fig:my_label}
\end{figure}
  

Para3
\section{Problem Definition}
\label{sec:problem}
         Para 1

\chapter{Literature Survey}
\section{Paper 1}
Para1

\section{Paper 2}
Para 2
\vskip 10cm

			
\chapter{Software Requirements Specification}
\section{Scope}
\begin{itemize}
\item \textbf{BACKGROUND TO THE PROPOSED WORK}:-\newline Developers in the field of IT play a pivotal role in the development of the field since, they are responsible for designing innovative and feasible applications that benefit the target customers. But most of their time is spent on surfing through the Internet for syntax related queries or understanding snippets written by other developers, rather than in developing the application logic itself. Thus, in this paced environment, it becomes essential for creating a software that aids the developers by not only providing quick syntax for various languages using their voice, but also enables them to understand code written by others by providing a succinct description of the code. 
\item \textbf{OBJECTIVE/PURPOSE}:- \newline To create a voice-based assistant for aiding developers in providing code snippet summarization and voice-based query execution within 5 months.   
\item \textbf{DELIVERABLES}:-\newline1.To create a voice-based assistant for interaction with clients. \newline2.To generate high level, english-interpretable summary of a given code snippet for the user. \newline3.Execute terminal commands for the user based on voice input.\newline4.Creating JAVA-based documentation for a given JAVA code snippet.\newline5.Integration with Github to provide voice-based creating, branching, forking, etc of repositories.\newline 6.Integration with StackOverflow for providing concise, and time efficient solution to specific code related queries.
\item \textbf{ASSUMPTIONS}:-\newline1.Microphone available on the machine on which the application is run.\newline2.Clients interacting with the assistant are aware of the basic sentence structures of English language.
\end{itemize}
\\



\section{Functional Requirements}  
\subsection{FR1}
\subsubsection{Description and Priority}
Content
\subsubsection{Functional Requirements}
Content

\subsection{FR2}
\subsubsection{Description and Priority}
Content
\subsubsection{Functional Requirements}
Content




\section{External Interface Requirements }  
\subsection{User Interfaces}
Content
\section{Nonfunctional Requirements }  

\subsection{Performance Requirements}

\subsection{Safety Requirements}

\subsection{Security Requirements}

\subsection{Software Quality Attributes}


\section{System  Requirements } 
\subsection{Database Requirements}

\subsection{Minimum Software and Hardware Requirements}
\begin{itemize}
    \item Python 3
    \item Torch framework
    \item Google's DialogFlow framework
    \item Suitable API's for integration with StackOverflow, Github, etc
    \item Text Editor/ Integrated Development Environment(IDE)
    \item ANTLR v4 (Another tool for Language Recognition) for JAVA parsing
    \item Ubuntu 16.04 or above
    \item Graphic Processing Unit(GPU)
    \item Intel i5 Processor 7th Generation
    \item 8GB RAM
\end{itemize}

\subsection{Hardware Requirements}


\section{Analysis Model} 


\section{System Implementation Plan} 
\begin{figure}[h]
    \centering
    \includegraphics[width=450pt]{Gantt.png}
    \caption{Gantt chart}
    \label{fig:my_label}
\end{figure}

\chapter{System Design}
\section{System Architecture}
\begin{figure}[h]
    \centering
     \includegraphics[width=430pt]{architecture.png}
    \caption{System architecture}
    \label{fig:my_label}
\end{figure}


\newpage
\section{Data Flow Diagrams}
 \subsection{DFD Level 0}
 \begin{center}
     \centering
     \includegraphics[width=400pt]{dfdzero.png}
     \captionof{figure}{DFD Level 0}
     \label{fig:my_label}
 \end{center}
    
   
\subsection{DFD Level 1}
\vskip 1cm
Given message recommend users
\begin{figure}[h]
    \centering
\includegraphics[width=400pt]{dfdone.png}    \caption{DFD Level 1(Message)}
    \label{fig:my_label}
\end{figure}
    
\subsection{DFD Level 1}
\vskip 1cm
Given user recommend messages
\begin{figure}[h]
    \centering
\includegraphics[width=400pt]{dfdone_.png}    \caption{DFD Level 1(User)}
    \label{fig:my_label}
\end{figure}
   
 
 \section{UML Diagrams}
 \subsection{UseCase Diagram}
\begin{center}
    \centering
\includegraphics[width=300pt,height=500pt]{usecase.png}
\captionof{figure}{Use Case Diagram}
    \label{fig:my_label}
\end{center}
  
 \subsection{Activity Diagram}
 \begin{center}
     \centering
\includegraphics[width=450pt,height=520pt]{activity.png}
\captionof{figure}{Activity Diagram}
     \label{fig:my_label}
 \end{center}
  
\subsection{Communication Diagram}
\begin{center}
    \centering
    \includegraphics[width=450pt]{comm.png}
    \captionof{figure}{Communication Diagram}
    \label{fig:my_label}
\end{center}
   
\subsection{Sequence Diagram}
\begin{center}
    \centering
    \includegraphics[width=450pt]{sequence.png}
    \captionof{figure}{Sequence Diagram}
    \label{fig:my_label}
\end{center}
      



\chapter{Other Specification}
\section{Advantages}
\begin{itemize}
	\item Item
    \item Item
    \item Item
    \item Item
\end{itemize}




\section{Limitations}
\begin{itemize}
	\item Item 
    \item Item
\end{itemize}


\section{Applications}
\begin{itemize}
	\item Item
    \item Item
\end{itemize}







\chapter{Project Plan}

\section{Project Estimates}

\subsection{Reconciled Estimates}
\subsubsection{Cost Estimate}
\begin{enumerate}
	\item Cost of Computer Systems with NVIDIA GPU cards

\end{enumerate}

\subsubsection{Time Estimates}
\begin{enumerate}
	\item Finalizing the Problem Statement: July 2018
	\item Designing: August 2018
	\item Implementation: October 2018 - December 2018
	\item Testing: January 2019
	\item Deployment: February 2019
\end{enumerate}

\subsection{Project Resources}
\begin{enumerate}
	\item Hardware:
	\begin{enumerate}
		\item NVIDIA GPU systems
		
	\end{enumerate}
	\item Software:
	\begin{enumerate}
		\item Gonuts dataset
		\item Pytorch
	\end{enumerate}
\end{enumerate}
%Project resources  [People, Hardware, Software, Tools and other resources] based on Memory Sharing, IPC, and Concurrency derived using appendices to be referred. 

 \chapter{Conclusion and future scope}
\section{Conclusion}
Content

% \bibliographystyle{plain}

\bibliographystyle{ieeetr}
\bibliography{biblo}

\begin{appendices}

\chapter{Feasibility Study}
\section{Problem statement feasibility}
\-\hspace{0.5in}Para 1
\-\hspace{0.5in}Para 2

\section{Mathematical model}

Let S be the solution perspective of given problem statement.
\\
\\
\centerline{S = $\{s, e, X, Y, F, DD, NDD, Su, Fa | \emptyset_s\}$}
\\
\\
Where,\\
s : Start state\\
s = $\{U,T\}$\\
\-\hspace{1cm}where, \\
\-\hspace{1cm}U = \\
\-\hspace{1cm}T = \\
\\
e : End state\\
e = \\
\\
X = $\{U,T,G\}$\\
\-\hspace{1cm}where, \\
\-\hspace{1cm}U = \\
\-\hspace{1cm}T = \\
\\
\\
Y = $\{M\}$\\
\-\hspace{1cm}where, \\
\-\hspace{1cm}M = 
\\
\\
F : set of functions\\
F = $\{F1, F2, F3, F4, F5\}$\\
F1 : \\
F2 : \\
F3 : \\
F4 : \\
F5 : \\
\\
DD : Deterministic Data\\
DD = $\{U, T, G\}$\\
NDD : \\
NDD = $\{M\}$\\\\
Su :\\
Fa : \\



\chapter{Summary of papers}
  \textbf{[1] Paper name}
   \\
   Description\\
\\

\\
\\

\chapter{Plagiarism Report}

\begin{figure}[h]
    \centering
    \includegraphics[width=400pt]{report.png} 
    \caption{Plagiarism Report}
    \label{fig:my_label}
\end{figure}
    


\end{appendices}

\chapter{References}
\begin{enumerate}
\item [[ 1]] Paper name\\

\end{enumerate}

\end{document}