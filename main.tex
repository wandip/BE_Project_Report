%This is a very basic  BE PROJECT PRELIMINARY template.

%############################################# 
%#########Author : Dipak Shantanu Tanmay Tejas###########
%#########COMPUTER ENGINEERING############


\documentclass[oneside,a4paper,12pt, times]{book}
\usepackage[left=1.5in, right=1in, top=1in, bottom=1.25in]{geometry}
\onehalfspacing
\setlength{\parskip}{12pt}
%\usepackage{showframe}
%\hoffset = 8.9436619718309859154929577464789pt
%\voffset = 13.028169014084507042253521126761pt
\usepackage{fancyhdr}

\fancypagestyle{plain}{%
  \fancyhf{}
  \fancyfoot[CE]{Pune Institute of Computer Technology, Department of Computer Engineering 2016-17}
  \fancyfoot[C]{\thepage}
}
\pagestyle{plain}
\renewcommand{\headrulewidth}{0pt}
\footskip = 0.625in
\cfoot{}
\rfoot{}

\usepackage[]{hyperref}
\usepackage{tikz}
\usetikzlibrary{arrows,shapes,snakes,automata,backgrounds,petri}
\usepackage{titlesec}
\usepackage{tabularx}

\usepackage[nottoc,notlot,notlof,numbib]{tocbibind}
\usepackage[titletoc]{appendix}
\usepackage{titletoc}
\renewcommand{\appendixname}{Annexure}
\renewcommand{\bibname}{References}

\setcounter{secnumdepth}{5}

\usepackage{float}
\usepackage{subcaption}
\usepackage{multirow}
\usepackage{biblatex}
\usepackage{url}
\addbibresource{CODA.bib}
\usepackage[ruled,vlined]{algorithm2e}

\begin{document}

\setlength{\parindent}{0mm}
\begin{center}

 \vspace*{1\baselineskip}
{\bfseries A PRELIMINARY REPORT ON \\}
 \vspace*{2\baselineskip}
{\bfseries \fontsize{16}{12} \selectfont  Comprehensive Developer Assistant (CODA) \\ \vspace*{2\baselineskip}}
{\fontsize{12}{12} \selectfont SUBMITTED TO THE SAVITRIBAI PHULE PUNE UNIVERSITY, PUNE \\
IN THE PARTIAL FULFILLMENT OF THE REQUIREMENTS \\
FOR THE AWARD OF THE DEGREE \\ 
\vspace*{2\baselineskip}
OF\\

\vspace*{2\baselineskip}}
{\bfseries \fontsize{14}{12} \selectfont \mbox{BACHELOR OF ENGINEERING(Computer Engineering)} \\
\vspace*{1\baselineskip}} 
{\bfseries \fontsize{14}{12} \selectfont SUBMITTED BY \\ 
\vspace*{1\baselineskip}} 
Shantanu Jain  \hspace{25 mm} Exam No : B150054307\\
Tanmay Singhal \hspace{25 mm} Exam No : B150054452 \\
Tejas Srivastava   \hspace{25 mm} Exam No : B150054460\\
Dipak Wani \hspace{25 mm} Exam No : B150054476\\
\vspace*{2\baselineskip}

\vskip 1cm
\includegraphics[width=100pt]{college_logo.jpg} \\
\vskip 1cm
{\bfseries \fontsize{15}{12} \selectfont 
DEPARTMENT OF COMPUTER ENGINEERING\\
}
{\bfseries \fontsize{14}{12} \selectfont{
\vspace*{1\baselineskip}
PUNE INSTITUTE OF COMPUTER TECHNOLOGY \\
\vspace*{1\baselineskip}
DHANKAWADI, PUNE - 43 \\
\vspace*{1\baselineskip}

}
\end{center}

\newpage

{\bfseries \fontsize{14}{12} \selectfont \centerline{SAVITRIBAI PHULE PUNE UNIVERSITY}
\vspace*{1\baselineskip}} 
{\bfseries \fontsize{14}{12} \selectfont \centerline{2018-2019}
\vspace*{1\baselineskip}} 
\begin{figure}[ht]
\centering
\includegraphics[width=100pt]{college_logo.jpg}
\end{figure}





{\bfseries \fontsize{16}{12} \selectfont \centerline{CERTIFICATE} 
\vspace*{2\baselineskip}} 

\centerline{This is to certify that the Project Entitled}
\vspace*{.5\baselineskip} 


{\bfseries \fontsize{14}{12} \selectfont \centerline{Comprehensive Developer Assistant (CODA)}
\vspace*{0.5\baselineskip}}

\centerline{Submitted by}
\vspace*{0.5\baselineskip} 
\centerline{Shantanu Jain  \hspace{25 mm} (B150054307)}
\centerline{Tanmay Singhal \hspace{25 mm} (B150054452)}
\centerline{Tejas Srivastava \hspace{25 mm}    (B150054460)}
\centerline{Dipak Wani \hspace{25 mm} (B150054476)}

is a bonafide work carried out by students under the supervision of \textbf{Prof. M.S. Chavan} and it
is submitted towards the partial fulfillment of the requirement of \textbf{Bachelor of Engineering (Computer Engineering)}.\\
\vskip 1cm
\bgroup
\def\arraystretch{0.7}
\begin{tabular}{c c }
\textbf{Prof. M.S. Chavan} &  \hspace{50 mm} \textbf{Dr. R. B. Ingle} \\								
Internal Guide   &  \hspace{50 mm} H.O.D \\
Dept. of Computer Engg.  &	\hspace{50 mm}Dept. of Computer Engg.  \\
\end{tabular}
%}
\begin{center}
%\fontsize{12}{18}\selectfont 
{
\vskip 1cm
\textbf{Dr. P. T. Kulkarni}\\
Principal\\
Pune Institute of Computer Technology  
}
\end{center}
\vskip 1cm
Signature of Internal Examiner \hspace{40 mm}\mbox{Signature of External Examiner}

\newpage

%\pictcertificate{TITLE OF BE PROJECT}{Student Name}{Exam Seat No}{Guide Name}
\setcounter{page}{0}

\frontmatter
\cfoot{PICT, Department of Computer Engineering 2018-19 \\ \\ \thepage}
%\rfoot{\thepage}
\pagenumbering{Roman}
%\pictack{BE PROJECT TITLE}{Guide Name}


{  \newpage {\bfseries \fontsize{14}{12} \selectfont \centerline{Acknowledgments} 
\vspace*{2\baselineskip}} \setlength{\parindent}{11mm} }
{ \setlength{\parindent}{0mm} }
%Please Write here Acknowledgment.Example given as\\
\textit{It gives us great pleasure in presenting the preliminary project report 
on {\bfseries \fontsize{12}{12} \selectfont `{\bfseries \fontsize{14}{12} \selectfont {Comprehensive Developer Assistant (CODA)}
}'}.}


 \textit{We would like to take this opportunity to thank our internal guide
 \textbf{Prof. M.S. Chavan} for giving us all the help and guidance we needed. We are really grateful to him for his kind support. His valuable suggestions were very helpful.} 
\\
\\
 \textit{We are also grateful to \textbf{Dr. R. B. Ingle}, Head of Computer
 Engineering Department, PICT for his indispensable
 support along with his guidance.}

\\
\textit{In the end our special thanks to the college for
providing various resources such as  laboratory with all needed software platforms,
continuous Internet connection, for our Project.}
\vspace*{3\baselineskip} \\
\begin{tabular}{p{8.2cm}c}
&Shantanu Jain\\
&Tanmay Singhal\\
&Tejas Srivastava\\
&Dipak Wani\\
&(B.E. Computer Engg.)
%}
\end{tabular}



{  \newpage {\bfseries \fontsize{14}{12} \selectfont \centerline{Abstract} 
\vspace*{2\baselineskip}} \setlength{\parindent}{11mm} }
{ \setlength{\parindent}{10mm} }

 \-\hspace{0.5in}\textit{Robots and AI based systems are now slowly taking over many sectors such as support,
industrial automation, etc. Our aim is to develop one such intelligent engine, which will
allow users to give commands in natural English to perform a particular task/s on the client
machine, which would then be analyzed and converted into appropriate terminal
commands/code snippet generation using Natural Language Processing and Machine
Learning. System can be integrated with VCS like GitHub, Project management application
like Slack. The system will be able to constantly evolve and train itself to handle variations
in user input. Such a system finds useful application in a Virtual Intelligent Assistant,
which can be based on remote access and can be handled easily by naive users as well.
principles, to run on the client device}
\\ 
\\
\-\hspace{0.5in}

% \maketitle
\tableofcontents
\listoffigures



\mainmatter



\titleformat{\chapter}[display]
{\fontsize{16}{15}\filcenter}
{\vspace*{\fill}
 \bfseries\LARGE\MakeUppercase{\chaptertitlename}~\thechapter}
{1pc}
{\bfseries\LARGE\MakeUppercase}
[\thispagestyle{empty}\vspace*{\fill}\newpage]







\setlength{\parindent}{11mm}
\chapter{Introduction}

\section{Motivation}
\-\hspace{0.5in} Chatbots and virtual assistants represent a potential shift in how people interact with data and services online. Thus, they are a part of the evolution in user interface, which started initially with command line, moved over to GUI (Graphical User interface) and further now has moved on to Voice based Interfaces i.e. Chatbots. Chatbots are machine agents that serve as natural language user interfaces for data and service providers. Currently, chatbots are typically designed and developed for Mobile messaging applications. The current interest in chatbots is spurred by recent developments in artificial intelligence (AI) and machine learning . Chatbots are seen as a means for direct user or  customer engagement through text messaging for customer service or marketing purposes, bypassing the need for special-purpose apps or webpages.


But we propose to extend the functionality of chatbots and assistants for our everyday purposes as well. As computer engineering students, one of the major chunk of our daily routines is occupied by coding and development related tasks. Also, it is evident that a major part of development and coding related tasks is wasted in searching for small queries and doubts, often related to how things are to be done, on the internet rather than actual coding, thus reducing the efficiency of our work. Also, in our course of work, we waste a lot of time in performing many trivial but important tasks, such as uploading our work on Version Control Systems for collaboration with our colleagues on which we tend to waste a lot of time due to the interaction with the GUI involved. Another common obstacle frequently faced by developers is the problem in understanding the code written by other developers or problems pertaining to code readability, due to which a lot of time is wasted, and efficiency of coding is reduced.A recent survey conducted by us among our peers also helped us strengthen the above claims regarding the reduction in efficiency during coding due to the mentioned reasons. 


Hence with CODA, we propose to solve all the above mentioned problems currently being faced by developers and coders by incorporating features like voice based query resolver in integration with common doubt solving platforms like Stack Overflow, voice based execution of common terminal commands, integration with Version Control Systems and automatic code summarization capabilities. 
\section{Problem Definition}
\label{sec:problem}
        To develop a voice based assistant aiding developers and coders to increase the efficiency of their work by assisting in various trivial but consequential tasks like executing terminal commands, resolving code related queries by providing concise and time efficient solutions with integration with StackOverflow, automatic high level code summarization to understand code snippets, executing commands like creating, forking, branching respositories related to Version Control Systems like Github on voice input and creating Java based documentation for a given Java code snippet.    

\chapter{Literature Survey}
This Chapter presents works related to this project. Work has been done on individual aspects of the project as a whole. Ranking models and code summarizing systems have been developed but not under a single roof. Also they are built for distinct languages. 
\section{Paper 1}
StackOverflow mining for programming prompter\cite{monperrus} implements a IDE plugin that when given a context in the IDE, automatically gets pertinent discussions from Stack Overflow, evaluates their relevance, and, if a given confidence threshold is surpassed, notifies the developer about the available help. The Ranking model helps to evaluate its relevance.
\section{Paper 2}
Deep Code Search\cite{gu2018deep} discusses previously written code snippets by searching through a large-scale code base. Instead of matching text similarity, their approach jointly embeds code snippets and natural language descriptions into a high-dimensional vector space, in such a way that code snippet and its corresponding description have similar vectors.
\section{Paper 3}
Summarizing Source Code\cite{iyer2016summarizing} presents datadriven
approach for generating high level
summaries of source code. It presents the use of neural networks with attention to
produce sentences that describe code snippets and queries.The model is trained using corpus
collected from StackOverflow.
\section{Paper 4}
Automatic documentation generation \cite{mcburney2014automatic} proposes a technique that includes this context by analyzing
how the Java methods are invoked. Programmers benefit from the generated documentation
because it includes context information. This paper puts forth Natural Language Generation and Page Ranking methods for generating contextual document generation.
\vskip 10cm

			
\chapter{Software Requirements Specification}
\section{Scope}
\begin{itemize}
\item \textbf{BACKGROUND TO THE PROPOSED WORK}:-\newline Developers in the field of IT play a pivotal role in the development of the field since, they are responsible for designing innovative and feasible applications that benefit the target customers. But most of their time is spent on surfing through the Internet for syntax related queries or understanding snippets written by other developers, rather than in developing the application logic itself. Thus, in this paced environment, it becomes essential for creating a software that aids the developers by not only providing quick syntax for various languages using their voice, but also enables them to understand code written by others by providing a succinct description of the code. 
\item \textbf{OBJECTIVE/PURPOSE}:- \newline To create a voice-based assistant for aiding developers in providing code snippet summarization and voice-based query execution within 5 months.   
\item \textbf{DELIVERABLES}:-\newline1.To create a voice-based assistant for interaction with clients. \newline2.To generate high level, english-interpretable summary of a given code snippet for the user. \newline3.Execute terminal commands for the user based on voice input.\newline4.Creating Java-based documentation for a given Java code snippet.\newline5.Integration with Github to provide voice-based creating, branching, forking, etc of repositories.\newline 6.Integration with StackOverflow for providing concise, and time efficient solution to specific code related queries.
\item \textbf{ASSUMPTIONS}:-\newline1.Microphone available on the machine on which the application is run.\newline2.Clients interacting with the assistant are aware of the basic sentence structures of English language.
\end{itemize}
\\



\section{Functional Requirements}  
\subsection {Chatbot Capabilities}
\-\hspace{0.5in} The assistant should be capable enough to understand commands in natural english language and  respond to any input it receives. If the bot does not understand the input, it should ask for a more simplified input. If the bot understands the input, it should respond with the correct and appropriate  information. If the bot needs more information to find an answer, it should ask for more information. 

\subsection{Code Summarization}
\-\hspace{0.5in}



\subsection{External Interface Requirements }  
\subsection{User Interfaces}
Content









% --------------------------------------------------------------

% Non-Functional Requirements Starts Here
\section{Nonfunctional Requirements } 

% Performance Requirements start Here
\subsection{Performance Requirements}
\begin{itemize}
    \item
    \textbf{Response Time:} 
        Response time is expected to have 5 or less than 5 seconds as minimum 6s are required for attacker to enter in system.
    \item
    \textbf{Reliability:} 
	    System should be reliable enough to detect all kinds of breach in system in order to secure  the system.
    \item
    \textbf{Scalability:} 
		System should be able to extend the number of machines whenever required to extend.
    \item
    \textbf{Platform:} 
	    System should work on LINUX based systems.
\end{itemize}
% Performance Requirements end here

% Safety Requirements Start here
\subsection{Safety Requirements}
\begin{itemize}
    \item
    \textbf{Authorization:}
        System should allow only authorised users to check and alter the system status.
    \item
    \textbf{Data Protection:}
        Organization’s (actor’s) data should be private i.e. only accessible to lower tiers.
\end{itemize}
% Safety Requirements End here

% Security Requirements start here
\subsection{Security Requirements}
\begin{itemize}
    \item 
        System should secure itself by checking the breach in system itself.
\end{itemize}
% Security Requirements end here

% Software Quality Attributes Starts here
\subsection{Software Quality Attributes}
\begin{itemize}
    \item
        System should be able to handle by user with ease and quickly.
    \item 
        System output should be simple enough to be understood by everyone.
    \item
        System should be reliable and efficient enough for usage.
    \item
        Output should be readable.
    \item
        Software should be portable enough so that it can work on any LINUX based system without installing.
\end{itemize}
% Software Quality Attributes End here

% Non Functional Requirements End here

% ---------------------------------------------------------------












% ---------------------------------------------------------------

% System Requirements starts here

\section{System  Requirements } 

% Database Requirements starts here
\subsection{Database Requirements}
\begin{itemize}
    \item 
        MySQL database server for analytics
    \item 
        MongoDB for storing flow of conversations
\end{itemize}
% Database Requiremenst Ends here

% Minimum Software Requirements starts here
\subsection{Minimum Software Requirements}
\begin{itemize}
    \item 
        Python 3
    \item 
        Torch framework
    \item 
        Google's DialogFlow framework
    \item 
        Suitable API's for integration with StackOverflow, Github, etc
    \item 
        Text Editor/ Integrated Development Environment(IDE)
    \item 
        ANTLR v4 (Another tool for Language Recognition) for JAVA parsing
    \item 
        Ubuntu 16.04 or above
    \item 
        Tkinter and similar front-end GUI frameworks
\end{itemize}
% Minimum Software Requirements ends here

% Hardware Requirements starts here
\subsection{Hardware Requirements}
\begin{itemize}
    \item 
        Graphic Processing Unit(GPU)
    \item 
        Intel i5 Processor 7th Generation
    \item   
        8GB RAM
    \item   
        Microphone
    \item   
        Speaker
\end{itemize}
% Hardware Requirements ends here

% System Requirements ends here

% --------------------------------------------------------------









\section{Analysis Model} 


\section{System Implementation Plan} 
\begin{figure}[h]
    \centering
    \includegraphics[width=450pt]{Gantt.png}
    \caption{Gantt chart}
    \label{fig:my_label}
\end{figure}

\chapter{System Design}
\section{System Architecture}
\begin{figure}[h]
    \centering
     \includegraphics[width=430pt]{architecture.png}
    \caption{System architecture}
    \label{fig:my_label}
\end{figure}


\newpage
\section{Data Flow Diagrams}
 \subsection{DFD Level 0}
 \begin{center}
     \centering
     \includegraphics[width=400pt]{dfdzero.png}
     \captionof{figure}{DFD Level 0}
     \label{fig:my_label}
 \end{center}
    
   
\subsection{DFD Level 1}
\vskip 1cm
Given message recommend users
\begin{figure}[h]
    \centering
\includegraphics[width=400pt]{dfdone.png}    \caption{DFD Level 1(Message)}
    \label{fig:my_label}
\end{figure}
    
\subsection{DFD Level 1}
\vskip 1cm
Given user recommend messages
\begin{figure}[h]
    \centering
\includegraphics[width=400pt]{dfdone_.png}    \caption{DFD Level 1(User)}
    \label{fig:my_label}
\end{figure}
   
 
 \section{UML Diagrams}
 \subsection{UseCase Diagram}
\begin{center}
    \centering
\includegraphics[width=300pt,height=500pt]{usecase.png}
\captionof{figure}{Use Case Diagram}
    \label{fig:my_label}
\end{center}
  
 \subsection{Activity Diagram}
 \begin{center}
     \centering
\includegraphics[width=450pt,height=520pt]{activity.png}
\captionof{figure}{Activity Diagram}
     \label{fig:my_label}
 \end{center}
  
\subsection{Communication Diagram}
\begin{center}
    \centering
    \includegraphics[width=450pt]{comm.png}
    \captionof{figure}{Communication Diagram}
    \label{fig:my_label}
\end{center}
   
\subsection{Sequence Diagram}
\begin{center}
    \centering
    \includegraphics[width=450pt]{sequence.png}
    \captionof{figure}{Sequence Diagram}
    \label{fig:my_label}
\end{center}
      



\chapter{Other Specification}
\section{Advantages}
\begin{enumerate}
    \item Human effort reduction
    \begin{itemize}
        \item Trivial commands can be automated and thus user's time is saved. 
    \end{itemize}
    
    \item Beginner Friendly
    \begin{itemize}
        \item Beginners in coding can easily query the assistant and find the to their problems at one place.
        \item Integration with other tools helps inculcate better coding practices.
    \end{itemize}
    \item Deep Integration with work environments
    \begin{itemize}
        \item Integration with VCS, project management tools like slack helps in keeping track of the development of project. 
    \end{itemize}
\end{enumerate}

\section{Limitations}
\begin{enumerate}
   
    \item Quantifying code quality
    \begin{itemize}
        \item Different metrics judge different aspects of code and thus universal metric of judging the solution proposed is unavailable.
        \item Feedback from the user (Strongly Agree, Agree, Disagree) needs to be introduced which may be influenced by external factors
    \end{itemize}
    \item Dependence on external tools and API
    \begin{itemize}
        \item The system does not guarantee solution to every problem as it is dependent on external tools (stackoverflow/documentation) for the results.
        \item The answers from the web may be inaccurate and thus introduce fallacies in the system
    \end{itemize}
     \item Code Relevance
    \begin{itemize}
        \item The output returned from may have inaccuracies generated because of the ranking algorithm, which may rank 
    \end{itemize}
\end{enumerate}


\section{Applications}
\begin{enumerate}
    \item Beginners in coding: Time wasted in searching for trivial code can be reduced with a large margin
    \item Team project development: Integration with other working platforms helps in keeping track of the project better
    \item Open Source Projects: Documentation generation which proves to be the most tedious task can be automated 
\end{enumerate}







\chapter{Project Plan}

\section{Project Estimates}

\subsection{Reconciled Estimates}
\subsubsection{Cost Estimate}
\begin{enumerate}
	\item Cost of Computer Systems with NVIDIA GPU cards

\end{enumerate}

\subsubsection{Time Estimates}
\begin{enumerate}
	\item Finalizing the Problem Statement: July 2018
	\item Designing: August 2018
	\item Implementation: October 2018 - December 2018
	\item Testing: January 2019
	\item Deployment: February 2019
\end{enumerate}

\subsection{Project Resources}
\begin{enumerate}
	\item Hardware:
	\begin{enumerate}
		\item NVIDIA GPU systems
		
	\end{enumerate}
	\item Software:
	\begin{enumerate}
		\item Gonuts dataset
		\item Pytorch
	\end{enumerate}
\end{enumerate}
%Project resources  [People, Hardware, Software, Tools and other resources] based on Memory Sharing, IPC, and Concurrency derived using appendices to be referred. 

 \chapter{Conclusion and future scope}
\section{Conclusion}
Content

% \bibliographystyle{plain}

\bibliographystyle{ieeetr}
\bibliography{biblo}

\begin{appendices}

\chapter{Feasibility Study}
\section{Problem statement feasibility}
\-\hspace{0.5in}Para 1
\-\hspace{0.5in}Para 2

\section{Mathematical model}

Let S be the solution perspective of given problem statement.
\\
\\
\centerline{S = $\{s, e, X, Y, F, DD, NDD, Su, Fa | \emptyset_s\}$}
\\
\\
Where,\\
s : Start state\\
s = $\{U,T\}$\\
\-\hspace{1cm}where, \\
\-\hspace{1cm}U = \\
\-\hspace{1cm}T = \\
\\
e : End state\\
e = \\
\\
X = $\{U,T,G\}$\\
\-\hspace{1cm}where, \\
\-\hspace{1cm}U = \\
\-\hspace{1cm}T = \\
\\
\\
Y = $\{M\}$\\
\-\hspace{1cm}where, \\
\-\hspace{1cm}M = 
\\
\\
F : set of functions\\
F = $\{F1, F2, F3, F4, F5\}$\\
F1 : \\
F2 : \\
F3 : \\
F4 : \\
F5 : \\
\\
DD : Deterministic Data\\
DD = $\{U, T, G\}$\\
NDD : \\
NDD = $\{M\}$\\\\
Su :\\
Fa : \\



\chapter{Summary of papers}
  \textbf{[1] Paper name}
   \\
   Description\\
\\

\\
\\

\chapter{Plagiarism Report}

\begin{figure}[h]
    \centering
    \includegraphics[width=400pt]{report.png} 
    \caption{Plagiarism Report}
    \label{fig:my_label}
\end{figure}
    


\end{appendices}

\chapter{References}
\printbibliography[heading=none]

\end{document}
